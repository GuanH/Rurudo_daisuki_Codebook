% use xelatex

\documentclass[a4paper,10pt,twocolumn,oneside]{article}
\setlength{\columnsep}{10pt}                                                              %兩欄模式的間距
\setlength{\columnseprule}{0pt}                                                                %兩欄模式間格線粗細

\usepackage{amsthm}								%定義,例題
\usepackage{amssymb}
%\usepackage[margin=2cm]{geometry}
\usepackage{fontspec}								%設定字體
\usepackage{color}
\usepackage[x11names]{xcolor}
\usepackage{listings}								%顯示code用的
%\usepackage[Glenn]{fncychap}						%排版,頁面模板
\usepackage{fancyhdr}								%設定頁首頁尾
\usepackage{graphicx}								%Graphic
\usepackage{enumerate}
\usepackage{titlesec}
\usepackage{amsmath}
\usepackage[CheckSingle, CJKmath]{xeCJK}
% \usepackage{CJKulem}
%\usepackage[T1]{fontenc}
\titlespacing{\section}{0cm}{0cm}{0cm}
\titlespacing{\subsection}{0cm}{0cm}{0cm}
\usepackage{amsmath, courier, listings, fancyhdr, graphicx}
\topmargin=0pt
\headsep=5pt
\textheight=780pt
\footskip=0pt
\voffset=-40pt
\textwidth=545pt
\marginparsep=0pt
\marginparwidth=0pt
\marginparpush=0pt
\oddsidemargin=0pt
\evensidemargin=0pt
\hoffset=-42pt
\setmainfont [
    Path = .fonts/ttf/,
    UprightFont = *-Regular,
    BoldFont = *-Bold,
    ItalicFont = *-Italic
  ] {Consolas}

\setmonofont [        
    Path = .fonts/ttf/,
    UprightFont = *-Regular,
  ] {Monaco}


%\renewcommand\listfigurename{圖目錄}
%\renewcommand\listtablename{表目錄} 

%%%%%%%%%%%%%%%%%%%%%%%%%%%%%

\setCJKmainfont{微軟正黑體}

\XeTeXlinebreaklocale "zh"
\XeTeXlinebreakskip = 0pt plus 1pt				%設定段落之間的距離
\setcounter{secnumdepth}{3}						%目錄顯示第三層

%%%%%%%%%%%%%%%%%%%%%%%%%%%%%
\makeatletter
\lst@CCPutMacro\lst@ProcessOther {"2D}{\lst@ttfamily{-{}}{-{}}}
\@empty\z@\@empty
\makeatother
\lstset{											% Code顯示
language=C++,										% the language of the code
basicstyle=\footnotesize\ttfamily, 						% the size of the fonts that are used for the code
%numbers=left,										% where to put the line-numbers
numberstyle=\footnotesize,						% the size of the fonts that are used for the line-numbers
stepnumber=1,										% the step between two line-numbers. If it's 1, each line  will be numbered
numbersep=5pt,										% how far the line-numbers are from the code
backgroundcolor=\color{white},					% choose the background color. You must add \usepackage{color}
showspaces=false,									% show spaces adding particular underscores
showstringspaces=false,							% underline spaces within strings
showtabs=false,									% show tabs within strings adding particular underscores
frame=false,											% adds a frame around the code
tabsize=2,											% sets default tabsize to 2 spaces
captionpos=b,										% sets the caption-position to bottom
breaklines=true,									% sets automatic line breaking
breakatwhitespace=false,							% sets if automatic breaks should only happen at whitespace
escapeinside={\%*}{*)},							% if you want to add a comment within your code
morekeywords={*},									% if you want to add more keywords to the set
keywordstyle=\bfseries\color{Blue1},
commentstyle=\itshape\color{Red4},
stringstyle=\itshape\color{Green4},
}

%%%%%%%%%%%%%%%%%%%%%%%%%%%%%

\begin{document}
\pagestyle{fancy}
\fancyfoot{}
%\fancyfoot[R]{\includegraphics[width=20pt]{ironwood.jpg}}
\fancyhead[L]{NTOU owo}
\fancyhead[R]{\thepage}
\renewcommand{\headrulewidth}{0.4pt}
\renewcommand{\contentsname}{Contents} 

\scriptsize
\tableofcontents
%%%%%%%%%%%%%%%%%%%%%%%%%%%%%
\section{Basic}
\subsection{Linux 對拍}
\lstinputlisting{basic/stress_test_Linux.sh}
\subsection{Windows 對拍}
\lstinputlisting{basic/stress_test_Windows.bat}
\subsection{builtin 函數}
\lstinputlisting{basic/builtin_function.cpp}
\subsection{輸入輸出}
\lstinputlisting{basic/io.cpp}
\subsection{Python 輸入輸出}
\lstinputlisting{basic/python_input.py}

%%%%%%%%%%%%%%%%%%%%%%%%%%%%%
\section{Data Structure}
\subsection{Link-Cut Tree}
\lstinputlisting{datastructure/link_cut_tree.cpp}
\subsection{持久化線段樹}
\lstinputlisting{datastructure/persistent_seg_tree.cpp}
\subsection{Treap}
\lstinputlisting{datastructure/treap.cpp}
\subsection{zkw 線段樹}
\lstinputlisting{datastructure/zkw_seg_tree.cpp}
%%%%%%%%%%%%%%%%%%%%%%%%%%%%%
\section{Flow}
\subsection{Dinic}
\lstinputlisting{flow/dinic.cpp}
\subsection{匈牙利}
\lstinputlisting{flow/hungarian.cpp}
\subsection{KM}
\lstinputlisting{flow/km.cpp}
\subsection{MCMF}
\lstinputlisting{flow/mcmf.cpp}
%%%%%%%%%%%%%%%%%%%%%%%%%%%%%
\section{幾何}
\subsection{點宣告}
\lstinputlisting{geometry/pt_declare.cpp}
\subsection{矩形面積}
\lstinputlisting{geometry/area_of_rect.cpp}
\subsection{最近點對}
\lstinputlisting{geometry/closest_pair_of_points.cpp}
\subsection{凸包}
\lstinputlisting{geometry/convex_hull.cpp}
\subsection{兩直線交點}
\lstinputlisting{geometry/intersection_of_two_lines.cpp}
\subsection{兩線段交點}
\lstinputlisting{geometry/intersection_of_two_segments.cpp}
\subsection{李超線段樹}
\lstinputlisting{geometry/li_chao_seg_tree.cpp}
\subsection{最小包覆圓}
\lstinputlisting{geometry/minimal_circle.cpp}
\subsection{最小包覆球}
\lstinputlisting{geometry/minimal_ball.cpp}
\subsection{旋轉卡尺}
\lstinputlisting{geometry/rotating_caliper.cpp}
%%%%%%%%%%%%%%%%%%%%%%%%%%
\section{圖論}
\subsection{BCC}
\lstinputlisting{graph/bcc.cpp}
\subsection{重心剖分}
\lstinputlisting{graph/centroid_decom.cpp}
\subsection{歐拉路徑}
\lstinputlisting{graph/euler_path.cpp}
\subsection{極大團}
\lstinputlisting{graph/maximal_clique.cpp}
\subsection{最大團}
\lstinputlisting{graph/maximum_clique.cpp}
\subsection{SCC}
\lstinputlisting{graph/scc.cpp}
\subsection{SPFA}
\lstinputlisting{graph/spfa.cpp}
\subsection{差分約束}
\input{graph/system_of_differences.tex}
%%%%%%%%%%%%%%%%%%%%%%%%%%%%%%
\section{數論}
\subsection{離散根號}
\lstinputlisting{math/discrete_sqrt.cpp}
\subsection{ex-crt}
\lstinputlisting{math/excrt.cpp}
\subsection{ex-gcd}
\lstinputlisting{math/exgcd.cpp}
\subsection{FFT}
\lstinputlisting{math/fft.cpp}
\subsection{高斯消去法}
\lstinputlisting{math/gaussian_elim.cpp}
\subsection{反矩陣}
\lstinputlisting{math/inverse_matrix.cpp}
\subsection{喬瑟夫問題}
\lstinputlisting{math/josephus.cpp}
\subsection{定理}
\input{math/math_theorem.tex}
\subsection{Miller Rabin}
\lstinputlisting{math/miller_rabin.cpp}
\subsection{NTT}
\lstinputlisting{math/ntt.cpp}
\subsection{Pollard's Rho}
\lstinputlisting{math/pollard_rho.cpp}
%%%%%%%%%%%%%%%%%%%%%%%%%%%%%%%
\section{字串}
\subsection{KMP}
\lstinputlisting{string/kmp.cpp}
\subsection{馬拉車}
\lstinputlisting{string/manacher.cpp}
\subsection{回文樹}
\lstinputlisting{string/pal_tree.cpp}
\subsection{SA}
\lstinputlisting{string/sa.cpp}
\subsection{SAM}
\lstinputlisting{string/sam.cpp}
\subsection{樹哈希}
\lstinputlisting{string/tree_hash.cpp}
\subsection{trie}
\lstinputlisting{string/trie.cpp}
\subsection{Z-value}
\lstinputlisting{string/z_value.cpp}
%%%%%%%%%%%%%%%%%%%%%%%%%%%%%%%
\section{DP}
\subsection{數位 dp}
\lstinputlisting{dp/digit.cpp}
\subsection{SOS dp}
\lstinputlisting{dp/sos.cpp}
\subsection{p-median}
\lstinputlisting{dp/p_median.cpp}
%%%%%%%%%%%%%%%%%%%%%%%%%%%%%%%
\section{Other}
\subsection{黑魔法}
\lstinputlisting{other/black_magic.cpp}
\subsection{CDQ 分治}
\lstinputlisting{other/cdq.cpp}
\subsection{DLX}
\lstinputlisting{other/dlx.cpp}
\subsection{Hiber Curve}
\lstinputlisting{other/hibert_curve.cpp}
\subsection{霍夫曼編碼}
\lstinputlisting{other/huffman_code.cpp}
\subsection{模擬退火}
\lstinputlisting{other/simulated_annealing.cpp}
\end{document}